% !Mode:: "TeX:UTF-8"
\newproblem{t4-1}{\ifisflags \colorbox{black!20}{p\pageref{prob:t4-1} ,\ref{prob:t4-1}{\kai 题}}\hspace*{1ex}\fi \ifisyear (05,15 {\kai 分})\label{prob:t4-1}\fi 数字接收机如图\refa{fig:05-1-1}所示,设输入为等概分布的二进制单极性NRZ信号,幅度为1,输入等效噪声$n(t)$满足均匀分布、其概率密度函数为
#13#10$f_n(x)=\left\{
#13#10          \begin{array}{ll}
#13#10            1-|x|, & -1\leqslant x \leqslant 1 \
#13#10            0, & \text{其它}
#13#10          \end{array}
#13#10        \right.
#13#10$。试求:
#13#10\begin{enumerate}
#13#10  \item 最佳判决电平$V_T$。
#13#10  \item 最小误比特率$P_e$。
#13#10\end{enumerate}
#13#10\begin{figure}[htb]
#13#10    \centering
#13#10      \includegraphics[scale=1]{pics/05pics/05-1-1.pdf}
#13#10
#13#10    \caption{}
#13#10    \labela{fig:05-1-1}
#13#10  \end{figure}
}{
\begin{enumerate}
#13#10  \item 因为输入为等概分布的二进制单极性NRZ信号且幅度为1,所以令
#13#10  \[a_n=\left\{
#13#10          \begin{array}{ll}
#13#10            1, & y_{s_1}=1 \
#13#10            0, & y_{s_0}=0
#13#10          \end{array}
#13#10        \right.
#13#10  \],则最佳判决电平$V_T=\frac{y_{s_1}+y_{s_0}}{2}=\frac{1}{2}$。
#13#10  \item 最小误比特率:
#13#10  \begin{eqnarray*}
#13#10  % \nonumber to remove numbering (before each equation)
#13#10    P_e &=& P(a_n=0)\cdot\int^{+\infty}_{V_T}f(r|0)\,\mathrm{d}r+ P(a_n=1)\cdot\int^{V_T}_{-\infty}f(r|1)\,\mathrm{d}r\
#13#10     &=& \frac{1}{2}\int^{1}_{\frac{1}{2}}f_n(r-0)\,\mathrm{d}r+\frac{1}{2}\int_{0}^{\frac{1}{2}}f_n(r-1)\,\mathrm{d}r \
#13#10     &=&  \frac{1}{2}\int^{1}_{\frac{1}{2}}(1-r)\,\mathrm{d}r+\frac{1}{2}\int_{0}^{\frac{1}{2}}r\,\mathrm{d}r\
#13#10     &=& \frac{1}{8}
#13#10  \end{eqnarray*}
#13#10 \begin{parchment}[综述]
#13#10   \begin{enumerate}
#13#10      \item 这个题考察了接收系统的误码性能,要求求出最佳判决门限$V_T$和最小误比特率$P_e$,属于教材$P_{143}\sim P_{145}$的内容。
#13#10      \item 一般来说,发$a_n=0$时的误码概率密度$f(r|0)=\frac{1}{\sqrt{2\pi}\sigma_n}\mathrm{e}^{-\frac{r-y_{s_0}}{2\sigma_n^2}}$,发$a_n=1$时的误码概率密度$f(r|1)=\frac{1}{\sqrt{2\pi}\sigma_n}\mathrm{e}^{-\frac{r-y_{s_1}}{2\sigma_n^2}}$,其中的$\sigma_n^2$为噪声功率。发$a_n=0$时出错的条件概率为$P(\mathrm{err}|a_n=0)=P(r\geqslant V_T|a_n=0)=\int^{+\infty}_{V_T}f(r|0)\,\mathrm{d}r$,发$a_n=1$时出错的条件概率为$P(\mathrm{err}|a_n=1)=P(r< V_T|a_n=1)=\int_{-\infty}^{V_T}f(r|0)\,\mathrm{d}r$。 总的误码率
#13#10          \begin{eqnarray*}
#13#10          % \nonumber to remove numbering (before each equation)
#13#10           P_e  &=& P(a_n=1)\cdot P(\mathrm{err}|a_n=1)+P(a_n=0)\cdot P(\mathrm{err}|a_n=0) \
#13#10             &=&  P(a_n=1)\cdot\int^{V_T}_{-\infty}f(r|1)\,\mathrm{d}r+P(a_n=0)\cdot\int^{+\infty}_{V_T}f(r|0)\,\mathrm{d}r
#13#10          \end{eqnarray*}
#13#10          系统最佳判决门限$V_T=\frac{y_{s_1}+y_{s_0}}{2}+\frac{\sigma_n^2}{y_{s_1}-y_{s_0}}\ln\frac{P(a_n=0)}{P(a_n=1)}$。 当二进制序列等概取值时,$V_T=\frac{y_{s_1}+y_{s_0}}{2}$,且有$P_e=Q\left(\frac{y_{s_1}-y_{s_0}}{2\sigma_n}\right)=Q\left(\sqrt{\frac{(y_{s_1}-y_{s_0})^2}{4\sigma_n^2}}\right)$。
#13#10          \item 在本题中,误码概率密度函数不是通常的$f(r|0)$和$f(r|1)$,而是由题中给定的$f_n(x)$,因此计算$P_e$ 时,不能再套用$f(r|0)$和$f(r|1)$的表达式,而只能用题中给的$f_n(x)$的表达式。在2009年的第二题中考了类似的问题,方法完全一样。
#13#10    \end{enumerate}
#13#10 \end{parchment}
#13#10\end{enumerate}}

\newproblem{t3-1}{\ifisflags \colorbox{black!20}{p\pageref{prob:t3-1} ,\ref{prob:t3-1}{\kai 题}}\hspace*{1ex}\fi \ifisyear (05,15 {\kai 分})\label{prob:t3-1}\fi 发射机如图\refa{fig:05-2-1}所示,若输入测试信号为$m(t)=\cos(2\pi\times 10^3t)$,载波信号为$C(t)=\cos(2\pi\times 10^6t)$,带通滤波器的传递函数为
$|H(f)|=\left\{
          \begin{array}{ll}
            1, & 10^6+500\leqslant|f|\leqslant 10^6+1500 \            0, & \text{其它}
          \end{array}
        \right.
$,试求:
\begin{enumerate}
  \item 输出信号的时域表达式(正交表达式)。
  \item 输出信号的平均功率。
\end{enumerate}
\begin{figure}[htb]
    \centering
      \includegraphics[scale=1]{pics/05pics/05-2-1.pdf}

    \caption{}
    \labela{fig:05-2-1}
  \end{figure}
}{
\begin{enumerate}
  \item 由题,设输入带通滤波器的信号为$x_{in}(t)$,则有
  \begin{eqnarray*}
  % \nonumber to remove numbering (before each equation)
    x_{in}(t) &=& m(t)\cdot C(t) \     &=& \cos(2\pi\times10^3t)\cdot\cos(2\pi\times10^6t) \     &=& \frac{1}{2}[\cos(10^6+10^3)t+\cos2\pi(10^6-10^3)t]
  \end{eqnarray*}
  因为$x_{in}(t)$要通过传递函数为$|H(f)|=\left\{
          \begin{array}{ll}
            1, & 10^6+500\leqslant|f|\leqslant 10^6+1500 \            0, & \text{其它}
          \end{array}
        \right.$的带通滤波器。所以,输出的信号$s(t)=\frac{1}{2}\cos(10^6+10^3)t=\frac{1}{2}\cos(2\pi\times10^6t)\cdot\cos(2\pi\times10^3t)-\frac{1}{2}\sin(2\pi\times10^6t)\cdot\sin(2\pi\times10^3t)$
  \item 由(a)问知,输出信号的平均功率$P=(\frac{1}{2})^2\cdot\frac{1}{2}=\frac{1}{8}$。
\end{enumerate}
\begin{parchment}
  这个题考DSB调制和带通滤波器对信号频率的选通意义以及求简单的功率。
\end{parchment}}

