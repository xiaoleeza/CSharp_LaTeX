% !Mode:: "TeX:UTF-8"
\newproblem{t4-1}{\ifisflags \colorbox{black!20}{p\pageref{prob:t4-1} ,\ref{prob:t4-1}{\kai 题}}\hspace*{1ex}\fi \ifisyear (05,15 {\kai 分})\label{prob:t4-1}\fi 数字接收机如图\refa{fig:05-1-1}所示,设输入为等概分布的二进制单极性NRZ信号,幅度为1,输入等效噪声$n(t)$满足均匀分布、其概率密度函数为
$f_n(x)=\left\{
          \begin{array}{ll}
            1-|x|, & -1\leqslant x \leqslant 1 \            0, & \text{其它}
          \end{array}
        \right.
$。试求:
\begin{enumerate}
  \item 最佳判决电平$V_T$。
  \item 最小误比特率$P_e$。
\end{enumerate}
\begin{figure}[htb]
    \centering
      \includegraphics[scale=1]{pics/05pics/05-1-1.pdf}

    \caption{}
    \labela{fig:05-1-1}
  \end{figure}
}{
\begin{enumerate}
  \item 因为输入为等概分布的二进制单极性NRZ信号且幅度为1,所以令
  \[a_n=\left\{
          \begin{array}{ll}
            1, & y_{s_1}=1 \            0, & y_{s_0}=0
          \end{array}
        \right.
  \],则最佳判决电平$V_T=\frac{y_{s_1}+y_{s_0}}{2}=\frac{1}{2}$。
  \item 最小误比特率:
  \begin{eqnarray*}
  % \nonumber to remove numbering (before each equation)
    P_e &=& P(a_n=0)\cdot\int^{+\infty}_{V_T}f(r|0)\,\mathrm{d}r+ P(a_n=1)\cdot\int^{V_T}_{-\infty}f(r|1)\,\mathrm{d}r\     &=& \frac{1}{2}\int^{1}_{\frac{1}{2}}f_n(r-0)\,\mathrm{d}r+\frac{1}{2}\int_{0}^{\frac{1}{2}}f_n(r-1)\,\mathrm{d}r \     &=&  \frac{1}{2}\int^{1}_{\frac{1}{2}}(1-r)\,\mathrm{d}r+\frac{1}{2}\int_{0}^{\frac{1}{2}}r\,\mathrm{d}r\     &=& \frac{1}{8}
  \end{eqnarray*}
 \begin{parchment}[综述]
   \begin{enumerate}
      \item 这个题考察了接收系统的误码性能,要求求出最佳判决门限$V_T$和最小误比特率$P_e$,属于教材$P_{143}\sim P_{145}$的内容。
      \item 一般来说,发$a_n=0$时的误码概率密度$f(r|0)=\frac{1}{\sqrt{2\pi}\sigma_n}\mathrm{e}^{-\frac{r-y_{s_0}}{2\sigma_n^2}}$,发$a_n=1$时的误码概率密度$f(r|1)=\frac{1}{\sqrt{2\pi}\sigma_n}\mathrm{e}^{-\frac{r-y_{s_1}}{2\sigma_n^2}}$,其中的$\sigma_n^2$为噪声功率。发$a_n=0$时出错的条件概率为$P(\mathrm{err}|a_n=0)=P(r\geqslant V_T|a_n=0)=\int^{+\infty}_{V_T}f(r|0)\,\mathrm{d}r$,发$a_n=1$时出错的条件概率为$P(\mathrm{err}|a_n=1)=P(r< V_T|a_n=1)=\int_{-\infty}^{V_T}f(r|0)\,\mathrm{d}r$。 总的误码率
          \begin{eqnarray*}
          % \nonumber to remove numbering (before each equation)
           P_e  &=& P(a_n=1)\cdot P(\mathrm{err}|a_n=1)+P(a_n=0)\cdot P(\mathrm{err}|a_n=0) \             &=&  P(a_n=1)\cdot\int^{V_T}_{-\infty}f(r|1)\,\mathrm{d}r+P(a_n=0)\cdot\int^{+\infty}_{V_T}f(r|0)\,\mathrm{d}r
          \end{eqnarray*}
          系统最佳判决门限$V_T=\frac{y_{s_1}+y_{s_0}}{2}+\frac{\sigma_n^2}{y_{s_1}-y_{s_0}}\ln\frac{P(a_n=0)}{P(a_n=1)}$。 当二进制序列等概取值时,$V_T=\frac{y_{s_1}+y_{s_0}}{2}$,且有$P_e=Q\left(\frac{y_{s_1}-y_{s_0}}{2\sigma_n}\right)=Q\left(\sqrt{\frac{(y_{s_1}-y_{s_0})^2}{4\sigma_n^2}}\right)$。
          \item 在本题中,误码概率密度函数不是通常的$f(r|0)$和$f(r|1)$,而是由题中给定的$f_n(x)$,因此计算$P_e$ 时,不能再套用$f(r|0)$和$f(r|1)$的表达式,而只能用题中给的$f_n(x)$的表达式。在2009年的第二题中考了类似的问题,方法完全一样。
    \end{enumerate}
 \end{parchment}
\end{enumerate}
}

\newproblem{t3-1}{\ifisflags \colorbox{black!20}{p\pageref{prob:t3-1} ,\ref{prob:t3-1}{\kai 题}}\hspace*{1ex}\fi \ifisyear (05,15 {\kai 分})\label{prob:t3-1}\fi 发射机如图\refa{fig:05-2-1}所示,若输入测试信号为$m(t)=\cos(2\pi\times 10^3t)$,载波信号为$C(t)=\cos(2\pi\times 10^6t)$,带通滤波器的传递函数为
$|H(f)|=\left\{
          \begin{array}{ll}
            1, & 10^6+500\leqslant|f|\leqslant 10^6+1500 \            0, & \text{其它}
          \end{array}
        \right.
$,试求:
\begin{enumerate}
  \item 输出信号的时域表达式(正交表达式)。
  \item 输出信号的平均功率。
\end{enumerate}
\begin{figure}[htb]
    \centering
      \includegraphics[scale=1]{pics/05pics/05-2-1.pdf}

    \caption{}
    \labela{fig:05-2-1}
  \end{figure}
}{
\begin{enumerate}
  \item 由题,设输入带通滤波器的信号为$x_{in}(t)$,则有
  \begin{eqnarray*}
  % \nonumber to remove numbering (before each equation)
    x_{in}(t) &=& m(t)\cdot C(t) \     &=& \cos(2\pi\times10^3t)\cdot\cos(2\pi\times10^6t) \     &=& \frac{1}{2}[\cos(10^6+10^3)t+\cos2\pi(10^6-10^3)t]
  \end{eqnarray*}
  因为$x_{in}(t)$要通过传递函数为$|H(f)|=\left\{
          \begin{array}{ll}
            1, & 10^6+500\leqslant|f|\leqslant 10^6+1500 \            0, & \text{其它}
          \end{array}
        \right.$的带通滤波器。所以,输出的信号$s(t)=\frac{1}{2}\cos(10^6+10^3)t=\frac{1}{2}\cos(2\pi\times10^6t)\cdot\cos(2\pi\times10^3t)-\frac{1}{2}\sin(2\pi\times10^6t)\cdot\sin(2\pi\times10^3t)$
  \item 由(a)问知,输出信号的平均功率$P=(\frac{1}{2})^2\cdot\frac{1}{2}=\frac{1}{8}$。
\end{enumerate}
\begin{parchment}
  这个题考DSB调制和带通滤波器对信号频率的选通意义以及求简单的功率。
\end{parchment}
}

\newproblem{t6-1}{\ifisflags \colorbox{black!20}{p\pageref{prob:t6-1} ,\ref{prob:t6-1}{\kai 题}}\hspace*{1ex}\fi \ifisyear (05,10{\kai 分})\label{prob:t6-1}\fi PCM系统输入模拟信号低通信号的带宽为$4\,\mathrm{kHz}$均匀量化器的量化范围为$(-1,1)\,\mathrm{V}$,若每个采样值用$8\,\mathrm{Bit}$二进制码表示,试求:
\begin{enumerate}
  \item 传输该PCM信号所需的最小带宽。
  \item 在输入不过载且线路传输无误码的条件下,接收端恢复的模拟信号的峰值信噪比(信号峰值功率/噪声平均功率)。
  \item 若因传输误码使峰值信噪比比无误码时降低$13\,\mathrm{dB}$,求误码率$P_e$。
\end{enumerate}
}{
\begin{enumerate}
  \item 由题,模拟低通信号带宽$B=4\,\mathrm{kHz}$,由采样定理得PCM 采样频率$f_s=2B=8\,\mathrm{kHz}$,则PCM信号比特率$R_b=nf_s=8\times8000=64\,\mathrm{kbps}$

  设传输该PCM信号所需最小带宽为$B_T$,则由$R_b=2B_T$得$B_T=\frac{1}{2}R_b=32\,\mathrm{kHz}$。
  \item 峰值信噪比$\left(\frac{S}{N}\right)_{q-pk}=6.02n+4.77=6.02\times8+4.77=52.93\,\mathrm{dB}$。
  \item 设误码率为$P_e$,则由PCM传输系统的信噪比公式有:$\left(\frac{S}{N}\right)_{PCM-pk}=\frac{3M^2}{1+4(M^2-1)P_e}=52.93-13=39.93\,\mathrm{dB}\approx 40\,\mathrm{dB}$,代入$M=2^8$,解之,$P_e=7.12\times10^{-5}$。
\end{enumerate}
\begin{parchment}[综述]
  \begin{enumerate}
    \item PCM系统对输入的模拟信号离散化,对其进行抽样,量化编码处理。按模拟信号采样定理,PCM系统的采样频率$f_s$至少等于输入的模拟信号带宽的2倍,最后在信道中传输的PCM信号二进制比特流,其比特率$R_b=nf_s$,$n$为量化器的量化位数。PCM系统最后输出的是二进制的数字基带信号,设其通过的数字基带信号带宽为$B_T$,则要使码元彼此之间不存在码间串扰,必须使其比特率$R_b\leqslant 2B_T$。
    \item 相关的公式:
    \begin{enumerate}
      \item 均匀量化器的信噪比:$\left(\frac{S}{N}\right)_{q-dB}=6.02n+4.77+10\log_{10}\frac{S}{V^2}$,其中$S$为信号的功率,$V$则为量化器的最大量化值。
      \item 峰值信噪比:$\left(\frac{S}{N}\right)_{q-pk}=6.02n+4.77$。
      \item 平均信噪比:$\left(\frac{S}{N}\right)_{qAvr-pk}=6.02n$。
    \end{enumerate}
    \item PCM传输系统的信噪比:$\left(\frac{S}{N}\right)_{PCM}\frac{(S/N)_q}{1+4(M^2-1)P_b}$,且有$\left(\frac{S}{N}\right)_{PCM-pk}=\frac{3M^2}{1+4(M^2-1)P_b}$,$\left(\frac{S}{N}\right)_{PCM-Avr}=\frac{M^2}{1+4(M^2-1)P_b}$。其中,$M$为量化电平数,$M=2^n$,$P_b$为误比特率。
    \item 本题考PCM问题,是个高频考点,属于教材$P_{242}\sim P_{251}$的内容,已考过PCM问题的年份有05,06,08,11,12年。
  \end{enumerate}
\end{parchment}
}

\newproblem{x4-1}{\ifisflags \colorbox{black!20}{p\pageref{prob:x4-1} ,\ref{prob:x4-1}{\kai 题}}\hspace*{1ex}\fi \ifisyear (05,15{\kai 分})\label{prob:x4-1}\fi
\begin{enumerate}
  \item 已知$f(t)$如图\refa{fig:05-4-1}所示。画出$\frac{\mathrm{d}}{\mathrm{d}t}f(t)$的图形。若将$\frac{\mathrm{d}}{\mathrm{d}t}f(t)$表示为$\frac{\mathrm{d}}{\mathrm{d}t}f(t)=Ag(t-t_1)+Bg(t-t_2)$,其中$g(t)=\sum\limits^{\infty}_{k=-\infty}\delta(t-2k)$是冲激函数,确定$A$、$B$、$t_1$、$t_2$的值。
  \item 已知$x[n]=u[n]-u[n-4]$,画出其偶部分$\mathrm{Ev}\{x[n]\}$ 和奇部分$\mathrm{Od}\{x[n]\}$的图形。
\end{enumerate}
\begin{figure}[htb]
    \centering
      \includegraphics[scale=1]{pics/05pics/05-4-1.pdf}

    \caption{}
    \labela{fig:05-4-1}
  \end{figure}
}{
\begin{enumerate}
  \item 由$f(t)$信号的图形可得$\frac{\mathrm{d}}{\mathrm{d}t}f(t)$ 信号的图形如图\refa{fig:05-4-2}:
 \begin{figure}[htb]
    \centering
      \includegraphics[scale=1]{pics/05pics/05-4-2.pdf}

    \caption{}
    \labela{fig:05-4-2}
  \end{figure}
 可知,$\frac{\mathrm{d}}{\mathrm{d}t}f(t)=2\sum\limits^{+\infty}_{k=-\infty}\delta(t-2k)-2\sum\limits^{+\infty}_{k=-\infty}\delta(t-2k-1)$,因为$g(t)=\sum\limits^{+\infty}_{k=-\infty}\delta(t-2k)$,所以$\frac{\mathrm{d}}{\mathrm{d}t}f(t)=2g(t)-2g(t-1)$。则$A=2$,$B=-2$,$t_1=0$,$t_2=1$
  \item \begin{description}
          \item[\kai 因为] $x[n]=u[n]-u[n-4]$
          \item[\kai 所以] $\mathrm{Ev}\{x[n]\}=\frac{x[n]+x[-n]}{2}=\frac{u[n]-u[n-4]+u[-n]-u[-n-4]}{2}$\\ $\mathrm{Od}\{x[n]\}=\frac{x[n]-x[-n]}{2}=\frac{u[n]-u[n-4]-u[-n]+u[-n-4]}{2}$
        \end{description}
  %因为$x[n]=u[n]-u[n-4]$,所以,$\mathrm{Ev}\{x[n]\}=\frac{x[n]+x[-n]}{2}=\frac{u[n]-u[n-4]+u[-n]-u[-n-4]}{2}$,$\mathrm{Od}\{x[n]\}=\frac{x[n]-x[-n]}{2}=\frac{u[n]-u[n-4]-u[-n]+u[-n-4]}{2}$。
      $\mathrm{Ev}\{x[n]\}$与$\mathrm{Od}\{x[n]\}$的图形分别如图~\refb{fig:05-4-3}、图~\refb{fig:05-4-4}:
      \begin{figure}[ht]
\centering
\subfigure[]{%
\includegraphics[scale=1]{pics/05pics/05-4-3.pdf}
\labela{fig:05-4-3}}
\hspace{.5cm}
\subfigure[]{%
\includegraphics[scale=1]{pics/05pics/05-4-4.pdf}
\labela{fig:05-4-4}}
%
\caption{}
\labela{fig:05-4}
\end{figure}
\end{enumerate}
\begin{parchment}
  本题考查基本的作图能力和信号奇部与偶部的表达方法,属基本题型,由奥本海姆信号教材$P_{143}$习题$1.14$题改编而来。
\end{parchment}
}

